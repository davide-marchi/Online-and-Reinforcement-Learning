In this exercise we classify three MDPs---\textbf{RiverSwim}, \textbf{RiverSwim-2} (a modified version of RiverSwim), and the \textbf{4-room grid-world}---according to the following classes:
\begin{itemize}
    \item \textbf{Ergodic MDPs}: Every stationary policy induces a Markov chain that has a unique recurrent class (i.e., it is a unichain).
    \item \textbf{Communicating MDPs}: For every pair of states $s, s' \in \mathcal{S}$, there exists some policy under which $s'$ is reachable from $s$ (and vice versa).
    \item \textbf{Weakly Communicating MDPs}: The state space can be partitioned into a communicating set of recurrent states and a set of transient states.
\end{itemize}
Since an ergodic MDP is a special case of a communicating MDP, and every communicating MDP is weakly communicating, the classes satisfy
\[
\text{Ergodic} \subseteq \text{Communicating} \subseteq \text{Weakly Communicating}.
\]

Below, I analyze each MDP and provide counterexamples where needed.

\subsection*{(i) RiverSwim}

\paragraph{Ergodicity:}  
\textbf{Answer:} \texttt{False}.\\[0.5em]
\textit{Explanation:}  
Although the underlying transition graph of RiverSwim is (in principle) strongly connected (since for any two states one can design a policy to move from one to the other), there exist stationary policies that break the unichain property. For example, consider the policy that always selects the \emph{left} action. Under this policy the agent remains confined to the left-hand bank (state 1) and never reaches any other state. This shows that not every stationary policy yields a single recurrent class; hence, the MDP is not ergodic.

\paragraph{Communicating:}  
\textbf{Answer:} \texttt{True}.\\[0.5em]
\textit{Explanation:}  
By the structural properties of RiverSwim, for any pair of states $s$ and $s'$ one can construct a policy that uses the \emph{right} action (to move toward higher-indexed states) and the \emph{left} action (to move back) appropriately such that $s'$ is reachable from $s$. This mutual reachability implies that RiverSwim is a communicating MDP.

\paragraph{Weakly Communicating:}  
\textbf{Answer:} \texttt{True}.\\[0.5em]
\textit{Explanation:}  
Since being communicating implies that there exists at least one policy connecting every pair of states, the MDP is, in particular, weakly communicating.

\medskip

\subsection*{(ii) RiverSwim-2}

The modified MDP, RiverSwim-2, is defined as RiverSwim but with an extra state $s_{\text{extra}}$ that has two actions:
\begin{itemize}
    \item Under action $a_1$, the agent moves deterministically to state 1 (the left bank).
    \item Under action $a_2$, the agent moves to state 1 with probability $0.5$ or to state 2 with probability $0.5$.
\end{itemize}
All other transitions remain the same as in RiverSwim.

\paragraph{Ergodicity:}  
\textbf{Answer:} \texttt{False}.\\[0.5em]
\textit{Explanation:}  
Similar to the original RiverSwim, one can choose stationary policies (e.g., always taking an action that avoids exiting a subset of states) that do not visit all states. Thus, the unichain property does not hold.

\paragraph{Communicating:}  
\textbf{Answer:} \texttt{False}.\\[0.5em]
\textit{Explanation:}  
A key modification in RiverSwim-2 is that the extra state $s_{\text{extra}}$ is not reachable from any of the original RiverSwim states. While once in $s_{\text{extra}}$ the agent can move to state 1 (or 2), there is no action in the main chain that will take the agent to $s_{\text{extra}}$. Hence, there exist state pairs (namely, any state in the original set and $s_{\text{extra}}$) for which no policy can ensure mutual reachability. Therefore, RiverSwim-2 is not communicating.

\paragraph{Weakly Communicating:}  
\textbf{Answer:} \texttt{True}.\\[0.5em]
\textit{Explanation:}  
Even though the whole state space is not mutually reachable, the original states (the main chain) form a communicating recurrent class. The extra state $s_{\text{extra}}$ is transient because whenever it is visited the agent will eventually be forced into the communicating set. This structure fits the definition of a weakly communicating MDP.

\medskip

\subsection*{(iii) 4-room Grid-world}

The 4-room grid-world (often also called the frozen lake MDP in this context) has the following characteristics:
\begin{itemize}
    \item The agent has 4 actions (up, down, left, right) when not adjacent to walls.
    \item Due to a slippery floor, the chosen action results in moving in the intended direction with probability $0.7$, staying in the same state with probability $0.1$, or moving in one of the two perpendicular directions (each with probability $0.1$).
    \item Walls act as reflectors (i.e., if an action leads into a wall, the agent remains in the current state).
    \item Upon reaching the rewarding state (highlighted in yellow), the agent is teleported back to the initial state.
\end{itemize}

\paragraph{Ergodicity:}  
\textbf{Answer:} \texttt{False}.\\[0.5em]
\textit{Explanation:}  
Even though the grid is fully connected, there exist stationary policies (for instance, those that consistently choose actions that lead to self-loops—by always hitting the wall) that confine the agent to a strict subset of states. Thus, not every stationary policy produces a Markov chain with a single recurrent class.

\paragraph{Communicating:}  
\textbf{Answer:} \texttt{True}.\\[0.5em]
\textit{Explanation:}  
The underlying structure of the grid is such that for any two states there exists a policy (using the stochastic nature of the transitions and the teleportation mechanism) that connects them. In other words, even if some policies lead to degenerate behavior, one can always design a policy that makes every state reachable from any other state.

\paragraph{Weakly Communicating:}  
\textbf{Answer:} \texttt{True}.\\[0.5em]
\textit{Explanation:}  
Since the grid-world is communicating (in the structural sense described above), it is trivially weakly communicating.

\medskip

\subsection*{Summary Table}

\begin{table}[h]
\centering
\begin{tabular}{lccc}
\toprule
\textbf{MDP}              & \textbf{Ergodic} & \textbf{Communicating} & \textbf{Weakly Communicating} \\
\midrule
RiverSwim                & False          & True                & True \\
RiverSwim-2              & False          & False               & True \\
4-room grid-world        & False          & True                & True \\
\bottomrule
\end{tabular}
\caption{Classification of the MDPs based on their connectivity properties.}
\label{tab:mdp_classes}
\end{table}