We recall the setting: two finite discounted MDPs
\[
  M_1 \;=\; (S,A,P_1,R_1,\gamma)
  \quad\text{and}\quad
  M_2 \;=\;(S,A,P_2,R_2,\gamma),
\]
with the same discount factor $0<\gamma<1$ and finite state--action space. 
For each $(s,a)$ we have:
\[
  \bigl|R_1(s,a)-R_2(s,a)\bigr|\;\le\;\alpha,
  \quad
  \bigl\|P_1(\cdot\mid s,a)\;-\;P_2(\cdot\mid s,a)\bigr\|_1\;\le\;\beta,
  \quad
  R_1(s,a),\,R_2(s,a)\;\in\;[0,R_{\max}].
\]
Let $\pi$ be any fixed stationary policy (deterministic or randomized), and let 
$V^\pi_1,\;V^\pi_2$ denote its value functions in $M_1$ and $M_2$, respectively. 
We wish to show that, for every $s\in S$,
\[
  \bigl|V_1^\pi(s)\;-\;V_2^\pi(s)\bigr|
  \;\;\le\;\;
  \frac{\alpha\;+\;\gamma\,R_{\max}\,\beta}{(1-\gamma)^2}.
\]

\medskip
\noindent
\textbf{Bellman Operators.} 
Define the Bellman operator $T_m^\pi$ for each $M_m$ ($m=1,2$) by
\[
  (T_m^\pi V)(s)
  \;=\;
  \sum_{a\in A}\pi(a\mid s)
  \Bigl[\,R_m(s,a)\;+\;\gamma\,\sum_{s'}P_m(s'\mid s,a)\,V(s')\Bigr].
\]
Then $V_m^\pi$ is the unique fixed point: $V_m^\pi = T_m^\pi\,V_m^\pi$, i.e.
\[
  V_m^\pi(s)
  \;=\;
  (T_m^\pi V_m^\pi)(s)
  \;=\;
  \sum_{a}\pi(a\mid s)
  \Bigl[R_m(s,a)
        \;+\;\gamma\,\sum_{s'}P_m(s'\mid s,a)\,V_m^\pi(s')\Bigr].
\]

\medskip
\noindent
\textbf{Step 1: Decompose the difference.}  
For each $s\in S$, we have
\[
  V_1^\pi(s)\;-\;V_2^\pi(s)
  \;=\;
  \bigl(T_1^\pi V_1^\pi\bigr)(s)\;-\;\bigl(T_2^\pi V_2^\pi\bigr)(s).
\]
Add and subtract $\bigl(T_1^\pi V_2^\pi\bigr)(s)$ inside the absolute value:
\[
  \begin{aligned}
  \bigl|V_1^\pi(s)-V_2^\pi(s)\bigr|
  &=\;
  \bigl|\,T_1^\pi V_1^\pi(s)\;-\;T_2^\pi V_2^\pi(s)\bigr|
  \\
  &\le\;
  \underbrace{\bigl|\,T_1^\pi V_1^\pi(s)\;-\;T_1^\pi V_2^\pi(s)\bigr|}%
            _{\text{(1) same operator, diff in values}}
  \;+\;
  \underbrace{\bigl|\,T_1^\pi V_2^\pi(s)\;-\;T_2^\pi V_2^\pi(s)\bigr|}%
             _{\text{(2) same value, diff in operators}}
  \,.
  \end{aligned}
\]

\smallskip
\noindent
\textbf{(1) Same operator, difference in the value functions.}\\
Since $T_1^\pi$ has discount factor $\gamma$,
\[
  \bigl|\,T_1^\pi V_1^\pi(s) \;-\;T_1^\pi V_2^\pi(s)\bigr|
  \;=\;
  \gamma\;\sum_{a}\pi(a\mid s)\,
  \bigl|\sum_{s'}P_1(s'\mid s,a)\,V_1^\pi(s')
         -\sum_{s'}P_1(s'\mid s,a)\,V_2^\pi(s')
  \bigr|
  \;\le\;
  \gamma\,\max_{x}\bigl|\,V_1^\pi(x)-V_2^\pi(x)\bigr|.
\]
Define 
\[
  \Delta\;=\;\sup_{s\in S}\,\bigl|\,V_1^\pi(s)-V_2^\pi(s)\bigr|.
\]
Thus the first portion is at most $\gamma\,\Delta$.

\smallskip
\noindent
\textbf{(2) Same value function, difference in operators.}\\
Next,
\[
  \bigl|\,T_1^\pi V_2^\pi(s) \;-\;T_2^\pi V_2^\pi(s)\bigr|
  \;\;=\;\;
  \Bigl|\;
  \sum_{a}\pi(a\mid s)\,\bigl[R_1(s,a)+\gamma\!\!\sum_{s'}P_1(s'\mid s,a)\,V_2^\pi(s')\bigr]
  \;-\;
  \sum_{a}\pi(a\mid s)\,\bigl[R_2(s,a)+\gamma\!\!\sum_{s'}P_2(s'\mid s,a)\,V_2^\pi(s')\bigr]
  \Bigr|.
\]
We can group the terms:
\[
  \le\;
  \sum_{a}\pi(a\mid s)
  \Bigl|\,
    \underbrace{R_1(s,a)-R_2(s,a)}_{\le\,\alpha}
    \;+\;
    \gamma\!\sum_{s'}\bigl[P_1(s'\mid s,a)-P_2(s'\mid s,a)\bigr]\,V_2^\pi(s')
  \Bigr|.
\]
Hence
\[
  \bigl|\,T_1^\pi V_2^\pi(s) - T_2^\pi V_2^\pi(s)\bigr|
  \;\le\;
  \sum_{a}\pi(a\mid s)\Bigl[
    \alpha
    \;+\;
    \gamma\,\Bigl|\sum_{s'}\bigl[P_1(s'\mid s,a)-P_2(s'\mid s,a)\bigr]\,
         V_2^\pi(s')\Bigr|
  \Bigr].
\]
Since 
\[
  \sum_{s'} \bigl|P_1(s'\mid s,a)-P_2(s'\mid s,a)\bigr|
  \;\le\;
  \beta,
  \quad\text{and}\quad
  \bigl|V_2^\pi(s')\bigr|\;\le\;\frac{R_{\max}}{1-\gamma},
\]
it follows that
\[
  \Bigl|\sum_{s'}\bigl[P_1(s'\mid s,a)-P_2(s'\mid s,a)\bigr]\,
        V_2^\pi(s')
  \Bigr|
  \;\le\;
  \beta\,\frac{R_{\max}}{\,1-\gamma\,}.
\]
Hence
\[
  \bigl|\,T_1^\pi V_2^\pi(s) - T_2^\pi V_2^\pi(s)\bigr|
  \;\le\;
  \alpha \;+\;\gamma\,\beta\,\frac{R_{\max}}{1-\gamma}.
\]

\medskip
\noindent
\textbf{Combine (1) \& (2).}\\
Putting the two pieces together:
\[
  \bigl|V_1^\pi(s)-V_2^\pi(s)\bigr|
  \;\le\;
  \underbrace{\gamma\,\Delta}_{\text{from (1)}}
  \;+\;\underbrace{\alpha + \gamma\,\beta\,\frac{R_{\max}}{1-\gamma}}_{\text{from (2)}}.
\]
Therefore, taking supremum over $s\in S$:
\[
  \Delta 
  \;=\;
  \sup_{s}\bigl|\,V_1^\pi(s)-V_2^\pi(s)\bigr|
  \;\le\;
  \gamma\,\Delta \;+\; \alpha 
  \;+\;\gamma\,\beta\,\frac{R_{\max}}{\,1-\gamma\,}.
\]
Rearranging gives
\[
  (1-\gamma)\,\Delta 
  \;\le\;
  \alpha \;+\;\gamma\,\beta\,\frac{R_{\max}}{1-\gamma},
  \quad\Longrightarrow\quad
  \Delta
  \;\le\;
  \frac{\alpha}{\,1-\gamma\,}
  \;+\;
  \frac{\gamma\,\beta\,R_{\max}}{(1-\gamma)^2}.
\]
As in the usual derivations, since $\alpha/(1-\gamma)\,\le\,\alpha/(1-\gamma)^2$
whenever $0<\gamma<1$, one can in fact write
\[
  \Delta 
  \;\;\le\;\;  
  \frac{\alpha \;+\; \gamma\,\beta\,R_{\max}}{(1-\gamma)^2}.
\]
Thus, for every $s\in S$,
\[
  \bigl|\,V_1^\pi(s)\;-\;V_2^\pi(s)\bigr|
  \;\le\;
  \Delta
  \;\le\;
  \frac{\alpha \;+\;\gamma\,\beta\,R_{\max}}{(1-\gamma)^2}.
\]
This completes the alternate proof via Bellman equations.