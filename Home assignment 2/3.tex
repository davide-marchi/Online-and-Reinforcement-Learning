We provide here the results of the Monte Carlo simulation and exact computation of 
\(V^\pi\) for the RiverSwim MDP, where the policy \(\pi\) takes \emph{right} action 
with probability \(0.65\) in states \(\{1,2,3\}\) and always takes \emph{right} 
in states \(\{4,5\}\).

The code used to run the experiments is contained in 
\texttt{HA2\_RiverSwim.py}.

\medskip

\noindent
\subsection*{(i) Monte Carlo Estimation of \(\boldsymbol{V^\pi}\)}

\noindent
We generated \(n=50\) trajectories of length \(T=300\) each, for each possible start state, 
accumulating returns
\[
  G \;=\;\sum_{t=0}^{T-1}\gamma^{\,t}\,r_t,
\]
and then averaged over the simulated trajectories to obtain 
an approximate value \(V^\pi(s)\). 
We used the discount factor \(\gamma = 0.96\). 
Below are the resulting Monte Carlo estimates (in a few seconds of execution time):

\begin{table}[H]
  \centering
  \begin{tabular}{l|c|c}
  \hline
  \textbf{State} & \textbf{MC Estimate} & \textbf{Exact Value} \\
  \hline
  1 & 4.01793 & 4.12090 \\
  2 & 4.61536 & 4.71119 \\
  3 & 5.96556 & 6.33460 \\
  4 & 9.47802 & 9.73803 \\
  5 & 11.21253 & 11.17784 \\
  \hline
  \end{tabular}
  \caption{Comparison of Monte Carlo estimates vs. exact values.}
  \label{tab:mc-vs-exact}
\end{table}
  

\noindent
\subsection*{(ii) Exact Computation using the Bellman Equation}

\noindent
We also computed the exact value function 
\[
   V^\pi \;=\;\bigl(I - \gamma\,P^\pi\bigr)^{-1}\,r^\pi
\]
by constructing the transition matrix \(P^\pi\) and reward vector \(r^\pi\) under policy 
\(\pi\), and then numerically solving the linear system 
\(\,(I - \gamma\,P^\pi)\,v = r^\pi\) in Python with \texttt{numpy.linalg.solve}. 
The table above (right column) presents the resulting exact values of 
\(V^\pi(s)\) for \(s=1,\dots,5.\)

\medskip

Although the Monte Carlo estimates are not the finest approximation and slightly deviate 
from the exact values (particularly in states 3 and 4) due to limited sample size, 
they were obtained in just a few seconds of execution. Increasing the number of trajectories 
(or their length) would make the approximation even closer in practice.