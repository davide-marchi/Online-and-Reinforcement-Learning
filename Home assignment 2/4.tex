\emph{All of the Python code used to run these experiments (i.e.\ PI, VI, and Anchored VI, 
plus the 4-room environment and the visualizations) can be found in the file 
\texttt{HA2\_gridworld.py}.}

\bigskip

\noindent
\textbf{Environment Setup:}\\
We have a \(7\times7\) grid with walls, yielding 20 accessible states labeled
\(0\) to \(19\). State \(19\) (bottom-right corner) has reward \(1\) and 
is effectively absorbing once reached. The agent chooses from 4 compass actions 
(\(\mathtt{0}=\text{Up}\), \(\mathtt{1}=\text{Right}\), 
\(\mathtt{2}=\text{Down}\), \(\mathtt{3}=\text{Left}\)), but transitions are 
\emph{slippery}: probability \(0.7\) for the chosen direction, \(0.1\) each 
for perpendicular directions, and \(0.1\) for staying in place. 
We set \(\gamma=0.97\) by default, except in part (iii) where \(\gamma=0.998\).

\subsection*{(i) Solve the grid-world task using PI (Policy Iteration).}

\noindent
We implemented \emph{Policy Iteration}, which alternates:
\[
  \resizebox{\textwidth}{!}{
  $\text{Policy Iteration:}\quad
  \begin{cases}
    \text{(a) Evaluate current policy }\pi: 
      & V^\pi = (I - \gamma P^\pi)^{-1} \, r^\pi,\\
    \text{(b) Improve } \pi: & 
      \pi \;\gets\; \arg\max_{a} 
        \Bigl[r(s,a) + \gamma\sum_{s'}P(s'\!\mid s,a)\,V^\pi(s')\Bigr].
  \end{cases}$
  }
\]
It converged in 5 iterations. The resulting \emph{optimal policy} (array of size 20) is
\[
  \pi_{\mathrm{PI}} \;=\; 
  [\,1,\,1,\,1,\,2,\,2,\,2,\,0,\,2,\,3,\,2,\,2,\,1,\,1,\,1,\,1,\,2,\,1,\,0,\,1,\,0\,].
\]
The \emph{optimal value function} \(V^*(s)\), rounded to 2 decimals, is
\[
  \resizebox{\textwidth}{!}{
  $V^* \;=\;
  [\,23.07,\,23.97,\,25.15,\,26.26,\,25.40,\,23.97,\,23.07,\,27.71,\,26.40,\,25.15,\,
    29.12,\,26.26,\,27.71,\,29.12,\,30.40,\,31.74,\,25.40,\,26.40,\,31.74,\,33.33\,].$
  }
\]
We can visualize \(\pi\) on the \(7\times7\) grid as:
\[
\begin{array}{l}
\text{\ttfamily
[ [Wall,  Wall,  Wall,  Wall,  Wall,  Wall,  Wall],}\\
\quad \text{\ttfamily [Wall, Right, Right, Right, Down, Down, Wall],}\\
\quad \text{\ttfamily [Wall, Down,  Up,   Wall,  Down, Left,  Wall],}\\
\quad \text{\ttfamily [Wall, Down,  Wall, Wall,  Down, Wall,  Wall],}\\
\quad \text{\ttfamily [Wall, Right, Right, Right, Right, Down, Wall],}\\
\quad \text{\ttfamily [Wall, Right, Up,    Wall, Right, Up,   Wall],}\\
\quad \text{\ttfamily [Wall, Wall,  Wall,  Wall, Wall,  Wall,  Wall] ]}.
\end{array}
\]

\subsection*{(ii) Implement VI and use it to solve the grid-world task.}

\noindent
We then implemented \emph{Value Iteration}, repeatedly applying
\[
  V_{n+1}(s) 
   \;=\;\max_{a}\Bigl[r(s,a) + \gamma\sum_{s'}P(s'\mid s,a)\,V_{n}(s')\Bigr].
\]
With \(\gamma=0.97\), VI converged in 48 iterations and recovered \emph{the same} 
optimal policy and value function as in part (i). The same map display applies.

\subsection*{(iii) Repeat (ii) with \(\gamma=0.998\).}

\noindent
Now we let \(\gamma=0.998\).  VI converged more slowly (56 iterations). 
The \emph{value function} is larger (about 487--500 in some states), 
and the policy is slightly modified in some early states.  
We still see a path leading to state 19. 
Overall, it remains optimal, just at a bigger scale of values.

Regarding the increasing number of iterations to converge that is because when \(\gamma\) is larger (closer to 1), the updates contract the value function differences more slowly, so it typically takes \emph{more} iterations to reduce \( \|V_{n+1} - V_n\|_{\infty} \) below a given threshold. Intuitively, a higher \(\gamma\) means future rewards are weighted more heavily, causing bigger swings or slower settling of the estimated values across states, and thus lengthening the convergence process.


\subsection*{(iv) Anchored Value Iteration (Anc-VI) with different initial points.}

\noindent
We added an anchoring approach with:
\[
  V_{n+1} 
   \;=\;\beta_{n+1}\,V_0 \;+\;(1-\beta_{n+1})\,\max_{a}\{\dots\}.
\]
Using three different anchors:
\begin{enumerate}[label=(\alph*)]
\item $V_0=0$ \quad(\textbf{614 iterations to converge}),
\item $V_0= \mathbf{1}$ \quad(\textbf{613 iterations to converge}),
\item random $V_0$ in $[0,\,1/(1-\gamma)]^{nS}$ \quad(\textbf{606 iterations to converge}),
\end{enumerate}
they all converged to an \emph{optimal policy} as well, matching the same final $V^*$ observed before.  

\subsection*{(v) Compare the convergence speed of VI vs.\ Anc-VI (with the same anchor).}

\noindent
Finally, we compared standard VI \ (\texttt{anc=False}) and the anchored variant 
\ (\texttt{anc=True}) for each available initialization of the values of $V$ at time $0$:
\[
  \begin{array}{lccc}
  \toprule
    & \textbf{$V_0$ = 0} & \textbf{$V_0$ = 1} & \textbf{$V_0$ = Random} \\
  \midrule
  \textbf{Standard VI} & 592\,\text{iters} & 591\,\text{iters} & 583\,\text{iters}\\[5pt]
  \textbf{Anchored VI} & 614\,\text{iters} & 613\,\text{iters} & 606\,\text{iters}\\
  \bottomrule
  \end{array}
\]
In this experiment, the \emph{anchored} method needed slightly \emph{more} iterations 
than the standard VI.  However, they still converged to optimal policies. 
With different discount factors or parameters, anchored VI can sometimes speed up or slow down convergence.