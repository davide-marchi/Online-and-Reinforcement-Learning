\subsection{Multi-variate normal distribution}

In this exercise we use the notation 
\[
N(m,C)
\]
to denote the multivariate normal distribution with mean $m\in\mathbb{R}^n$ and covariance matrix $C\in\mathbb{R}^{n\times n}$. In particular, $N(0,I)$ denotes the standard normal distribution in $\mathbb{R}^n$. The notation here is consistent with that used in the Monte Carlo methods presentation (see \texttt{3.1 RL-MonteCarlo.pdf}).

\subsubsection*{1.}
Let $a\in\mathbb{R}^n$ be a nonzero vector and consider the matrix
\[
C = aa^T.
\]

\subsubsection*{(a) Rank of $C=aa^T$}
For any $x\in\mathbb{R}^n$ we have
\[
C x = aa^T x = a \, (a^T x).
\]
Since $a^T x$ is a scalar, it follows that $Cx$ is always a scalar multiple of $a$. In other words, the image (or column space) of $C$ is contained in $\operatorname{span}\{a\}$. Since $a\neq 0$, this is a one-dimensional subspace. Hence, 
\[
\operatorname{rank}(C)=1.
\]

\subsubsection*{(b) Eigenvector and Eigenvalue of $C=aa^T$}
We next show that $a$ is an eigenvector of $C$. Indeed,
\[
C\,a = aa^T a = a\,(a^T a) = \|a\|^2\,a.
\]
Thus, $a$ is an eigenvector corresponding to the eigenvalue 
\[
\lambda = \|a\|^2.
\]

\subsubsection*{(c) Maximum Likelihood for a One-Dimensional Normal Distribution}
Consider the family of one-dimensional normal distributions with zero mean and variance $\sigma^2$, that is, 
\[
N(0,\sigma^2).
\]
The probability density function (pdf) is given by
\[
p(a\mid \sigma^2)=\frac{1}{\sqrt{2\pi\sigma^2}}\exp\left(-\frac{a^2}{2\sigma^2}\right).
\]
For a single observation $a\in\mathbb{R}$, the likelihood function is
\[
L(\sigma^2)=p(a\mid \sigma^2)=\frac{1}{\sqrt{2\pi\sigma^2}}\exp\left(-\frac{a^2}{2\sigma^2}\right).
\]
It is more convenient to maximize the logarithm of the likelihood:
\[
\ell(\sigma^2)=\log L(\sigma^2)=-\frac{1}{2}\log(2\pi\sigma^2)-\frac{a^2}{2\sigma^2}.
\]
Differentiate $\ell(\sigma^2)$ with respect to $\sigma^2$:
\[
\frac{d\ell}{d\sigma^2}=-\frac{1}{2\sigma^2}+\frac{a^2}{2(\sigma^2)^2}.
\]
Setting the derivative equal to zero, we obtain
\[
-\frac{1}{2\sigma^2}+\frac{a^2}{2(\sigma^2)^2}=0 
\quad\Longrightarrow\quad \frac{a^2-\sigma^2}{2(\sigma^2)^2}=0.
\]
Thus,
\[
a^2-\sigma^2=0\quad\Longrightarrow\quad \sigma^2=a^2.
\]
This shows that the likelihood of generating $a\in\mathbb{R}$ is maximized when $\sigma^2=a^2$.