I focus on a scenario with:
\begin{itemize}
  \item \emph{Old product:} known success probability $0.5$.
  \item \emph{New product:} unknown success probability $\mu$.
\end{itemize}
At each time step $t=1,\ldots,T$, I must pick exactly one product to offer, aiming to maximize the total number of successful sales. Let $\Delta = 0.5 - \mu$:
\[
\Delta > 0 \quad \Longleftrightarrow \quad \mu < 0.5 \quad (\text{new product is worse}),
\]
\[
\Delta < 0 \quad \Longleftrightarrow \quad \mu > 0.5 \quad (\text{new product is better}).
\]
I propose the following procedure:

\subsection*{Proposed Strategy}

\paragraph{1) Initial $n$ Exploratory Steps on New Product.}
Choose a small fixed integer $n$ (for instance, $n=10$). In the first $n$ rounds, always offer the new product. Let $S_n$ be the total number of successes during these $n$ tries, so
\[
\hat{\mu}_n \;=\; \frac{S_n}{n}
\]
is an initial empirical estimate of the new product's success probability.

\paragraph{2) Bandit-style Indices (for $t>n$).}
From round $t=n+1$ onward, treat the problem like a 2-armed bandit:
\begin{itemize}
    \item \textbf{Arm 1 (Old Product)} has a known ``reward'' of $0.5$, so its index is simply $\text{Index}_{\text{old}} = 0.5$.
    \item \textbf{Arm 2 (New Product)} is updated with an empirical mean and a confidence bonus. Specifically, if by round $t-1$ I have tried the new product $N_{t-1}$ times in total, with $X_{t-1}$ successes, then
    \[
    \hat{\mu}_{t-1} \;=\; \frac{X_{t-1}}{N_{t-1}}, 
    \qquad
    \text{Index}_{\text{new}}(t) 
    \;=\;
    \hat{\mu}_{t-1}
    \;+\;
    \sqrt{\frac{2 \ln (t)}{2\,N_{t-1}}}.
    \]
\end{itemize}
At each step $t>n$, compare $\text{Index}_{\text{old}} = 0.5$ with $\text{Index}_{\text{new}}(t)$, and choose whichever is larger. Ties can be broken arbitrarily.

\subsection*{Pseudo-Regret Analysis}
Denote by $R_T$ the pseudo-regret up to time $T$, i.e.\ the difference between the expected number of successes of an optimal single choice (if $\mu$ were known) and that of my algorithm.

\paragraph{Case 1: $\mu > 0.5$ (new product is better).}
The best single choice is always picking the new product, with expected success $\mu$ each round. My algorithm invests $n$ steps initially in the new product; that part is no problem if the new product is better, because I am effectively collecting reward near $\mu$. After the first $n$ rounds, the bandit step begins, but the new product’s index is likely to exceed $0.5$ fairly soon. Indeed, once the empirical mean $\hat{\mu}_{t-1}$ stabilizes around $\mu>0.5$, the confidence bonus only makes the index bigger, ensuring that I select the new product almost every time. By standard bandit arguments, the number of times I might \emph{fail} to choose the new product (e.g.\ if it momentarily loses to $0.5$) is bounded by a constant (depending on $\mu-0.5$). Hence the pseudo-regret $R_T$ is $O(1)$ for $\mu>0.5$.

\paragraph{Case 2: $\mu < 0.5$ (new product is worse).}
Now the best single choice is always the old product. Initially, I still do $n$ test rounds with the new product, causing a regret on the order of $n\,(\,0.5 - \mu)$ from those forced tries. Then, in the subsequent bandit step, the UCB-based rule for the new product means I keep exploring it occasionally until I gather enough data to be confident that $\hat{\mu}_{t-1} + \sqrt{\frac{2\ln(t)}{2\,N_{t-1}}} < 0.5$. The standard analysis from \emph{UCB} (or \emph{LCB}) bandits says that the new product is pulled at most $O(\ln T)$ times if $\Delta = 0.5 - \mu$ is a positive gap. Summing the extra regret from each suboptimal pull yields $R_T = O(n + \ln T)$, but since $n$ is a fixed constant, that is effectively $O(\ln T)$.

\paragraph{Conclusion.}
Hence:
\[
R_T 
=\begin{cases}
O(1), & \text{if } \mu > 0.5,\\
O(\ln T), & \text{if } \mu < 0.5,
\end{cases}
\]
which satisfies the requirement that the pseudo-regret is constant in the better-than-old case, and logarithmic in the worse-than-old case.